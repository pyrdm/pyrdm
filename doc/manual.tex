%    Copyright (C) 2014 Christian T. Jacobs, Alexandros Avdis, Gerard J. Gorman, Matthew D. Piggott.

%    This file is part of PyRDM.

%    PyRDM is free software: you can redistribute it and/or modify
%    it under the terms of the GNU General Public License as published by
%    the Free Software Foundation, either version 3 of the License, or
%    (at your option) any later version.
%
%    PyRDM is distributed in the hope that it will be useful,
%    but WITHOUT ANY WARRANTY; without even the implied warranty of
%    MERCHANTABILITY or FITNESS FOR A PARTICULAR PURPOSE.  See the
%    GNU General Public License for more details.
%
%    You should have received a copy of the GNU General Public License
%    along with PyRDM.  If not, see <http://www.gnu.org/licenses/>.

\documentclass[a4paper,11pt]{report}
\usepackage[margin=1.25in]{geometry}
\usepackage{graphicx}

\setlength{\parskip}{0.25cm}
\setlength{\parindent}{0cm}

\begin{document}

\begin{titlepage}
\begin{center}
\vspace*{3cm}
\huge{PyRDM User Manual}\\\vspace*{2cm}
\includegraphics[width=0.4\columnwidth]{images/rdm.png}\\\vspace*{2cm}
\LARGE{Version 0.1}

\end{center}
\end{titlepage}

\tableofcontents

\chapter{Introduction}\label{chap:introduction}
\section{Overview}
PyRDM is a project that focuses on the relationship between data and the scientific software that has either created that data, or operates on that data to create new information. Two examples from geoscience are given for illustrative purposes:
\begin{itemize}
  \item Dispersion of saline solution into an estuary from desalination plant.
  \begin{itemize}
    \item Input data: Bathymetry, atmospheric forcing, etc.
    \item Software: Pre-/post-processing codes; numerical models.
    \item Output data: Provenance; simulated flow fields; diagnostic quantities.
  \end{itemize}
  \item Characterisation of pores media for carbon sequestration.
  \begin{itemize}
    \item Input data: Micro-CT images of rock samples.
    \item Software: Pre-/post-processing of data (e.g. image segmentation, mesh generation); numerical models.
    \item Output data: Provenance; computational meshes that could be used by other flow solvers; computed flow fields; diagnostics.
  \end{itemize}
\end{itemize}

From these two examples one can already start to see a great deal of generality emerging. In order to achieve reproducibility, or indeed computability, one must be able to capture all of the input data (i.e. collected/measured data which should also have its associated provenance). Next the actual software that draws information from that data must be captured, and care must be taken to ensure that the same version that created a particular result from the data is used. Finally, the output data from the software must be captured, including provenance data that details how the input data and software were used to create this new data. PyRDM aims to address these needs by facilitating the automated management and online publication of software and data via citable repositories hosted by Figshare. This library can be incorporated into the workflow of scientific software to allow research output to be curated in a straight-forward manner.

%The \texttt{pyrdm} directory contains a set of modules which can be imported into software-specific tools. The \texttt{bin} directory contains an example of such a tool, designed to publish the Fluidity CFD code and any associated data files to Figshare. Users wishing to 

\section{Licensing}
PyRDM is released under the GNU General Public License. Further details can be found in the COPYING file supplied with this software.

\chapter{Getting started}\label{chap:getting_started}
\section{System requirements}
A standard Python installation is required, as well as any additional Python modules that are listed in the README file under the ``Dependencies'' section. PyRDM is designed to run on the Linux operating system. All development and testing takes place on the Ubuntu Precise (12.04) distribution.

\section{Installation}
It is recommended that users use the terminal to install and run PyRDM. After navigating to the base directory of PyRDM (i.e. the directory that the Makefile is in), use the following command to install the PyRDM library:

  \texttt{make install}

\textbf{Note 1:} \texttt{sudo} may be needed for this if the default install directory is located outside of \texttt{/home}. This will yield a system-wide install of PyRDM, which is recommended.

\textbf{Note 2:} In order for Python to find the PyRDM module, you will need to add the PyRDM base directory to your \texttt{PYTHONPATH} environment variable, unless you have used \texttt{sudo} as mentioned in Note 1 above. This can be achieved using:

  \texttt{export PYTHONPATH=\$PYTHONPATH:/path/to/pyrdm}

You may wish to add this statement to your \texttt{/home/your\_username/.bashrc} or \texttt{/etc/bash.bashrc} files so the \texttt{PYTHONPATH} is set correctly each time you log in.

\section{Configuration}
You should copy the contents of the file \texttt{pyrdm.ini.example} to a new file called \texttt{pyrdm.ini} and save it in the \texttt{/home/your\_username/.config} directory. If this directory does not exist, please create it first using

  \texttt{mkdir /home/your\_username/.config}

The contents of the new file \texttt{pyrdm.ini} should then be modified as per the guidance in the following subsections.

\subsection{Figshare authentication}\label{sect:authentication}
PyRDM requires a set of authentication details in order to publish and modify files using your Figshare account. You will need to login and use the Figshare web interface to generate these authentication details, after which you should paste them into the \texttt{figshare} section of the configuration file.

\begin{enumerate}
  \item Go to \texttt{http://figshare.com/account/applications}
  \item Click \texttt{Create a new application}
  \item Fill in the application details as per Figure \ref{fig:application_details}
  \item Ensure that the application has read and write access to your Drafts, Private and Public articles, as per Figure \ref{fig:permissions}
  \item Click \texttt{Save changes}. PyRDM should appear in your list of applications which can access your account.
  \item Click \texttt{View} next to PyRDM and then click the \texttt{Access codes} tab to see the authentication details. The four fields should be pasted into the \texttt{pyrdm.ini} configuration file.
\end{enumerate}

\textbf{Note:} If you are publishing through a group account, you will need to ask the account's administrator for the authentication details.

\begin{figure}
  \centering
  \includegraphics[width=1\columnwidth]{images/application_details.png}
  \caption{The application details for PyRDM.}
  \label{fig:application_details}
\end{figure}

\begin{figure}
  \centering
  \includegraphics[width=1\columnwidth]{images/permissions.png}
  \caption{The set of permissions required by PyRDM.}
  \label{fig:permissions}
\end{figure}

\section{Testing}
PyRDM comes with a suite of unit tests which verify the correctness of its functionality. It is recommended that you run these unit tests before using PyRDM by executing:

  \texttt{make unittest}
  
on the command line. Many of these tests require access to a Figshare account, so please ensure that the \texttt{pyrdm.ini} setup file contains valid Figshare authentication tokens.

\chapter{PyRDM Functionality}

\section{Publishing software}
The publication of software is handled by the \texttt{publish\_software} method in the Publication class. This requires:

\begin{itemize}
  \item The Figshare authentication details (see Section \ref{sect:authentication}).
  \item The software's name.
  \item The version of the software that you would like to publish (for Git repositories, this is the SHA-1 commit hash).
  \item The location of the software's Git repository (or the location of any file within that repository) on your local hard drive.
  \item The ID of a category in Figshare's categories list. The full list of categories can be found here: \texttt{http://api.figshare.com/v1/categories}.
\end{itemize}

\subsection{Author attribution}
If an AUTHORS file is provided in the Git repository's base directory, PyRDM parses it and looks for strings of the form \texttt{figshare:xxxx}, where \texttt{xxxx} is an author ID. Author IDs should be specified after each author's full name. An example is: 

\texttt{Christian Jacobs (figshare:554577)}

PyRDM automatically adds all authors who provide their Figshare author IDs to the software publication.

\section{Publishing data}
The publication of software is handled by the \texttt{publish\_software} method in the Publication class. This requires:

\begin{itemize}
  \item The Figshare authentication details (see Section \ref{sect:authentication}).
  \item A dictionary of parameters, containing the following key-value pairs:
    \begin{itemize}
      \item \texttt{title}: the title of the dataset
      \item \texttt{description}: a description of the dataset
      \item \texttt{files}: a list of paths to the files within the dataset.
      \item \texttt{category\_id}: the ID of a category in Figshare's categories list
      \item \texttt{tag\_name}: a single string, or list of strings, to tag the data's fileset with
    \end{itemize}
  \item Optionally, an \texttt{article\_id} if the dataset already exists on the Figshare servers and you wish to update it. By default, this is set to \texttt{None}.
\end{itemize}

\subsection{MD5 cross-checks}
When a data file is published, the file's MD5 checksum is stored in a corresponding checksum file. The next time the user tries to publish the file, its MD5 checksum is recomputed and compared against the MD5 checksum stored in its corresponding MD5 file. If the two MD5 checksums are different (or the checksum file does not exist), the file is uploaded to the Figshare server and the checksum file is updated with the new checksum. If the two checksums are the same, the file is unmodified and is not re-uploaded. This can help prevent unnecessary bandwidth usage and is particularly useful when you have large data files which are not frequently modified.

\chapter{Application: Fluidity-Publish}
Fluidity-Publish is a Python program which uses the PyRDM library. It is designed specifically for use with the Fluidity computational fluid dynamics code, located at \texttt{https://github.com/FluidityProject/fluidity}. Currently, you must use a fork of Fluidity which contains the publishing functionality. This is available here:

\texttt{https://github.com/ctjacobs/fluidity-rdm}

but will soon be merged into Fluidity's master branch. Note that you must build Fluidity (or at least satisfy the libspud dependency noted in the README file) before using Fluidity-Publish. You may need to add Fluidity's 'python' directory to your \texttt{PYTHONPATH} environment variable in order for the libspud module to be found.

\section{Enable publishing}
If we want to publish the Fluidity source code used to run a particular simulation, along with the input and output data, we first need to enable the `publish' option in that simulation's configuration/options file (the ``.flml'' file), as shown in Figure \ref{fig:diamond}. Simply select which publishing service you wish to use (e.g. Figshare or Zenodo), and under the \texttt{input\_files} and \texttt{output\_files} sub-options (see Figure \ref{fig:diamond}), enter the paths (relative to the options file) to the input and output files you wish to publish in the following format (a Python list of strings): 

\texttt{["path/to/file1", "path/to/file2", "path/to/file3"]}

You may use wildcard characters here (e.g. ``*vtu'' to publish all VTK-based simulation output). Instead of creating a new code repository or fileset on Figshare or Zenodo, you may also publish to an existing one by entering its ID in the following option(s):

\texttt{/publish/\{software,input\_data,output\_data\}/article\_id}

\begin{figure}
  \centering
  \includegraphics[width=1\columnwidth]{images/diamond.png}
  \caption{The `publish' option enabled in a simulation's configuration file. The file is being modified in Diamond.}
  \label{fig:diamond}
\end{figure}

\section{Using Fluidity-Publish}
To publish, you will need to use the \texttt{fluidity-publish} program (located in PyRDM's 'bin' directory) at the command line; this expects one or more of the following options, followed by the (relative or absolute) path to the simulation's configuration file:
\begin{itemize}
   \item \texttt{-s} : Publish the Fluidity source code. This can be used in conjunction with the \texttt{-v} option to explicitly choose the version of Fluidity (in the form of the Git SHA-1 hash of a particular commit) that you want to publish.
   \item \texttt{-i} : Publish the input data files whose paths are specified in the options file.
   \item \texttt{-o} : Publish the output data files whose paths are specified in the options file.
   \item \texttt{-p} : Publish the software or data, but keep it private. Must be used in conjunction with the \texttt{-s}, \texttt{-i} or \texttt{-o} option. Note that any DOI generated will not be valid until the publication is made public.
\end{itemize}

e.g. \texttt{fluidity-publish -i /data/fluidity/tests/top\_hat\_cg\_supg/top\_hat\_cg\_supg.flml}

or, if you did not install PyRDM, change directory (\texttt{cd}) to the PyRDM base directory 

(e.g. \texttt{cd /home/my\_username\_here/pyrdm}) and use:

\texttt{python bin/fluidity-publish -i /data/fluidity/tests/top\_hat\_cg\_supg/top\_hat\_cg\_supg.flml}

Note that the software, input data and output data must be published separately. You cannot yet use the -s, -i and -o options together.

Once the publication process has finished, the ID and DOI of the publication will be added to the simulation's configuration file for future reference. If you wish to publish the simulation data again, the ID and DOI will be re-used (unless you remove them from the configuration file).

\subsection{Software version}
Unless you provide a particular version of Fluidity at the command-line using the \texttt{-v} option, Fluidity-Publish will automatically obtain the version of Fluidity from the file \texttt{version.h} stored in the \texttt{include} directory of the local Fluidity repository on your computer. This file is created at compile-time when the Fluidity binary is built. If this file is not present (perhaps because you haven't built Fluidity yet), then Fluidity-Publish will instead use the version (SHA-1 key) of the HEAD commit of the local repository.

\subsection{Provenance data}
Fluidity writes a limited amount of provenance data to the header of the simulation's `stat' file. If you choose to publish the output data (which should include the `stat' file) using the \texttt{-o} option, then Fluidity-Publish will (if available) retrieve the IDs and DOIs of the recently published software and input data from the simulation's options file. It will then add those to the existing provenance data before publishing the output data files that you have specified.


\chapter{Application: PyRDM-Publish}
Another publishing tool (see \texttt{bin/pyrdm-publish}) has been created to publish PyRDM itself to Figshare. At the moment this is only used as an example program to demonstrate PyRDM's functionality without the need to install libspud (or Fluidity and libspud) as a dependency, and therefore any Figshare publications produced using this tool will be kept private in your Figshare account. You can visit the "My data" page on Figshare to view the publication details.

Only the source code hosted on PyRDM's GitHub repository can be published; the publishing tool does not publish any input/output files. To use \texttt{pyrdm-publish}, you will need to provide the path to the local Git repository on your computer containing PyRDM. For example,

\texttt{pyrdm-publish -i /home/my\_username\_here/pyrdm}

(or \texttt{python /path/to/pyrdm/bin/pyrdm-publish /path/to/pyrdm}, if you did not install PyRDM).

\end{document}
